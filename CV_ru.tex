\documentclass[12pt,a4paper,sans]{moderncv}
\usepackage{savesym}
\savesymbol{fax}
\usepackage{marvosym}
\restoresymbol{MARV}{fax}
\moderncvstyle{classic}
\moderncvcolor{green}

\usepackage[russian]{babel}
\usepackage{color}
\usepackage[scale=0.87]{geometry}

\setlength{\hintscolumnwidth}{3.5cm}
\renewcommand*{\namefont}{\fontsize{26}{28}\mdseries\upshape}

\name{Михаил Юрьевич}{Туник}
\begin{document}
\makecvtitle

\vspace*{-12mm}
\begin{tabular}{ c }
\Letter~Почта: mihael.8112@yahoo.com
\end{tabular}

\section{Образование}
\cventry{2013~--- 2017}{бакалавр}{Санкт-Петербург}{Санкт-Петербургский политехнический университет Петра Великого}{кафедра прикладной математики и механики}{}
\cventry{2017~--- 2019}{магистр}{Санкт-Петербург}{Санкт-Петербургский политехнический университет Петра Великого}{кафедра прикладной математики и механики}{}

\subsection{Магистерская работа}
\cventry{2019}{Исследование ядерной оценки плотности вероятности в условиях малой выборки}{}{}{}{}

Работа посвящена исследованию теоретической точности статистической ядерной оценки специального типа в случае конечного размера выборки.

\section{Опыт работы: 3 года 4 месяца}
\cventry{август 2019~--- настоящее время}{Санкт-Петербургский государственный университет, междисциплинарная исследовательская лаборатория им. П.Л.Чебышева}{программист-математик}{}{}{}
\begin{itemize}
\item Работа в команде над созданием статистического инструмента для анализа геоданных с помощью гауссовских процессов (multi-output GP, sparse GP) на Python.
Добавление новой функциональности, рефакторинг существующей кодовой базы. Разбор научных статей в предметной области.\newline
\item Работа над созданием ПО для решения задач сейсмической инверсии. Разработка подхода на основе разработанного на предыдущем этапе инструмента.
Использование фреймворков Tensorflow/Torch. Участие в разработке, проведение тестов разработанного метода на реальных данных. \newline
\item Разработка ПО для решения задач Римана, возникающих в ходе решения уравнений гидродинамики в пористых средах.
Поиск и обработка литературы, проектирование алгоритмов и их реализация. Написание библиотеки на C++ для использования в проекте на Python через ctypes. \newline
\item Подбор параметров сложных численных симуляций задач гидродинамики в среде Dumux с помощью байесовской оптимизации, библиотека botorch. Участие в алгоритмической разработке инструмента.
Дополнительно: работа с готовыми дашбордами экспериментов (Tensorboard),
проектирование и разработка собственного интерфейса ML-системы (PyQt5).
\end{itemize}

\section{Навыки}
\cvitem{Для научных расчетов: }{Python (numpy, scipy, keras, autograd), R, Mathematica}{}{}{}{}
\cvitem{Математический бэкграунд: }{математическая статистика и теория вероятностей (в том числе случайные функции: процессы, поля), теория интегральных преобразований, линейная алгебра, дискретная математика.}{}{}{}{}
\cvitem{Алгоритмическая подготовка:\hspace*{5mm}}{стандартный курс алгоритмов и численных методов, комбинаторная оптимизация, статистический анализ данных, машинное обучение: основные подходы к регрессии, классификация по табличным данным.}{}{}{}{}

\cvitem{Дополнительно: }{
\begin{itemize}
\item Большой опыт работы с Linux [Ubuntu, Mint], настройка системы и работа через bash; \newline
\item Опыт работы с фреймворками GPFlow, GPy для работы с гауссовскими процессами, небольшой опыт работы с CatBoost; навыки работы с Pandas Dataframes, sklearn;\newline
\item Свободная работа с LaTeX для написания математических текстов и презентаций; \newline
\item Опыт создания устанавливаемых пакетов для Python; \newline
\item Опыт создания оконных приложений на PyQt5, небольшой опыт сборки бинарных файлов для релиза через PyInstaller; \newline
\item Система контроля версий Git, управление репозиториями в Bitbucket и GitHub (pull-requests, code review и т.п.); \newline
\item Запуск кода на удаленном сервере через ssh, сборка и настройка Docker на базовом уровне, работа через virtual environment с Python проектами; \newline
\item C/C++ на среднем уровне (в том числе параллельные вычисления с OpenMP, написание make-файлов, создание небольших .so библиотек, CMake); \newline
\item Работа с PostgreSQL на базовом уровне, pgAdmin, libpq++.
\end{itemize}
}{}{}{}{}
\section{Иностранные языки}
\cvitemwithcomment{Английский}{Upper-Intermediate}{}{}
\end{document}
