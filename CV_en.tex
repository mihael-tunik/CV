\documentclass[12pt,a4paper,sans]{moderncv}
\usepackage{savesym}
\savesymbol{fax}
\usepackage{marvosym}
\restoresymbol{MARV}{fax}
\moderncvstyle{classic}
\moderncvcolor{green}

\usepackage[russian]{babel}
\usepackage{color}
\usepackage[scale=0.87]{geometry}

\setlength{\hintscolumnwidth}{3.5cm}
\renewcommand*{\namefont}{\fontsize{26}{28}\mdseries\upshape}

\name{Mihael}{Tunik}
\begin{document}
\makecvtitle

\vspace*{-12mm}
\begin{tabular}{ c }
\Letter~Почта: mihael.8112@yahoo.com
\end{tabular}

\section{Education}
\cventry{2013~--- 2017}{Bachelor degree}{Saint-Petersburg}{Peter the Great St.Petersburg Polytechnic University}{departament of applied mathematics and mechanics}{}
\cventry{2017~--- 2019}{Master degree}{Saint-Petersburg}{Peter the Great St.Petersburg Polytechnic University}{departament of applied mathematics and mechanics}{}

\subsection{Master thesis}
\cventry{2019}{Special kernel density estimator for finite sample size conditions}{}{}{}{}
Work is dedicated to research of theoretical accuracy of statistical kernel density estimator of special type for finite sample size conditions.

\section{Experience: 3 years 2 months}
\cventry{august 2019~--- now}{Saint-Petersburg State University, Chebyshev Laboratory}{engineer-researcher}{}{}{}
\begin{itemize}
\item Work in team on develomment of special statistical instrument for geo-data analysis based on Gaussian Processes (multi-output GP, sparse GP) written mostly on Python language. Adding new features, refactoring of existing codebase. Research for relevant scientific articles in given subject area.\newline
\item Project work on software for solving inverse problems for seismic data. Developing specific approach based on previously developed software.
Using Tensorflow/Torch frameworks. Participation in development and testing on real data. \newline
\item Development of software for solving Riemann problems, which appear in porous media hydrodynamics.
Search for literature and articles, algorithm develompment and implementation. Created library on C++ for using in Python project via Ctypes. \newline
\item Fine-tuning advanced hydrodynamic simulations in Dumux with Bayesian Optimization technics, using botorch. Participation in development of original algorithm and implementation.
Also work with experimental dashboards like Tensorboard and building custom UI for developed ML-system with PyQt5.
\end{itemize}

\section{Skills}
\cvitem{For scientific computation: }{Python (numpy, scipy, keras, autograd), R, Mathematica}{}{}{}{}
\cvitem{Mathematical background: }{statistics and probability theory (random functions and fields), linear algebra, calculus.}{}{}{}{}
\cvitem{Computer science background:\hspace*{5mm}}{Standard course of algorithms and numerical methods, various optimization methods, statistical data analisys, ML: regression of all types, table data classification.}{}{}{}{}

\cvitem{More information and keywords: }{
\begin{itemize}
\item Decent 5-year experience with Linux [Ubuntu, Mint], system configuration, work via bash; \newline
\item Experience with GPFlow, GPy for work with Gaussian Processes, gradient boosting with CatBoost; work skills with Pandas Dataframes and sklearn;\newline
\item Advanced work with LaTeX for texts and presentations; \newline
\item Experience with building up python package from scratch; \newline
\item Experience with PyQt5, and also with PyInstaller for bulding binaries; \newline
\item Git version control system, managing repositories in Bitbucket and GitHub (pull-requests, code review and so on); \newline
\item Experience with running code on servers remotely via ssh, building Docker containers, work with everything via virtual environments; \newline
\item Some experience with C/C++ (parallel computations with OpenMP, make-files and CMake, building small .so libs); \newline
\item Basic knowledge of PostgreSQL (including pgAdmin and libpq++).
\end{itemize}
}{}{}{}{}
\section{Languages}
\cvitemwithcomment{Russian}{Native speaker}{}{}
\cvitemwithcomment{English}{Upper-Intermediate}{}{}
\end{document}
